\subsection{Capitolato C6: \textit{Showroom 3D}}

\subsubsection{Informazioni generali}

\textbf{Proponente:} \textit{Sanmarco Informatica S.p.A.}

\textbf{Descrizione:} Il capitolato richiede la creazione di uno showroom virtuale per presentare dei prodotti, farsi conoscere e vendere alla clientela, avvalendosi dell’ambiente e dell’esperienza immersiva offerta.

Questo ambiente virtuale consentirebbe inoltre alta flessibilità e riduzione dei costi allo showroom aziendale.

\subsubsection{Dominio}

\textbf{Dominio applicativo}

Il capitolato richiede la creazione di un ambiente tridimensionale navigabile con mouse e tastiera, dove poter esporre una collezione di elementi vendibili in un ambiente limitato.

\textbf{Tecnologie}

Per lo sviluppo dell'applicazione viene consigliato l'uso di Three.js o in alternativa Unity o Unreal Engine. In generale è necessario conoscere una tecnologia per lo sviluppo 3D, non è richiesta però l'integrazione con il VR.

\subsubsection{Fattori determinanti}
\begin{itemize}
    \item Complessità elevata del progetto;
    \item ambienti di sviluppo;
    \item capitolato molto lungo.
\end{itemize}

\subsubsection{Decisione}

L'ambito applicativo del capitolato ci è sembrato allo stesso tempo affascinante e complesso, la sua difficoltà e la poca esperienza con le tecnologie da utilizzare ci ha portato però a decidere di \textbf{rifiutare il progetto.}
