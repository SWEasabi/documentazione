\subsection{Capitolato C2: \textit{Lumos Minima}}
\subsubsection{Informazioni generali}
\textbf{Proponente:} \textit{Imola Informatica S.p.A.}

\textbf{Descrizione:} Ci viene richiesto di sviluppare un progetto che aiuti nell'attuale sfida di ridurre i costi e lo spreco di energia.

L'ottimizzazione proposta integra l'illuminazione pubblica con il mondo dell'IoT.

L'applicazione che ci impegnamo a sviluppare dovrà permettere ai comuni di adattare l'illuminazione alle varie situazioni che si verranno a verificare. Queste situazioni saranno per esempio: l'adattamento della luminosità a seconda dell'entità presente vicino ad un corpo illuminante.

Obiettivo non di minor importanza sarà integrare nella soluzione una modalità di rilevamento e gestione dei guasti. 

\subsubsection{Dominio}

\textbf{Dominio applicativo}

Il capitolato vuole sviluppare una applicazione web responsive dotata delle seguenti funzioni:
\begin{itemize}
    \item login e logout di un operatore;
    \item collegamento di un impianto ai dati derivanti da un sensore;
    \item gestione manuale di un impianto luminoso;
    \item aumento o riduzione globale dell'intensità luminosa (da operatore o tramite dati di sensore);
    \item aumento o riduzione locale dell'intensità luminosa (da operatore o tramite dati di sensore);
    \item permettere all'operatore di inserire manualmente un guasto;
    \item rilevazione automatica di anomalie tramite sensori.
\end{itemize}

\textbf{Tecnologie}

Per lo sviluppo dell'applicazione è necessario conoscere e usare:
\begin{itemize}
    \item MQTT o un qualsiasi altro protocollo di comunicazione consueto nel mondo IoT;
    \item microservizi;
    \item REST;
    \item una tecnologia per costruire i vari sistemi di coordinamento e gestione, per esempio Python.
\end{itemize}



\subsubsection{Fattori determinanti}
\begin{itemize}
    \item Chiarezza del capitolato e disponibilità del referente;
    \item progetto green, attuale e futuro;
    \item competenze richieste vicine a quelle possedute del gruppo.
\end{itemize}

\subsubsection{Decisione}
Il capitolato ci ha fin da subito colpito molto per la sua tematica stimolante e improntata al futuro ma utile già dal presente prefissandosi di risolvere tematiche legate al risparmio energetico.

La libertà lasciatoci a livello di organizzazione e scelta delle tecnologie è stato punto cardine che ci ha spinti a \textbf{scegliere questo capitolato}.
