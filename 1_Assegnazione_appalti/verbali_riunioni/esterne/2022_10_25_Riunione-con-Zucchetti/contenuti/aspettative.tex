\section{Punti del giorno e aspettative}
    \subsection{Sostituire CAPTCHA in una maniera più accessibile o replicare dei CAPTCHA già esistenti?}
        \begin{itemize}
            \item Il progetto può essere approcciato in diversi modi:
            \begin{itemize}
                \item Prendere un CAPTCHA già esistente;
                \item reinventarlo completamente.
            \end{itemize}
            \item l'accessibilità non è considerata importante.
        \end{itemize}
    \subsection{Manualistica}
        \begin{itemize}
            \item La manualistica sarà revisionata solo dal professore;
            \item La discussione invece sarà importante per il proponente.
        \end{itemize}
    \subsection{Incontri e comunicazioni}
        \begin{itemize}
            \item Incontri solo quando necessario, non a cadenze programmate.
        \end{itemize}
    \subsection{Ci devono essere testing o requisiti per il coverage?}
        \begin{itemize}
            \item Testing e coverage interessano solo al professore.
        \end{itemize}
    \subsection{Vanno eseguite verifiche di sicurezza?}
        \begin{itemize}
            \item Vanno eseguite severe verifiche di sicurezza tra server e client se utilizziamo una libreria già esistente;
            \item L'attenzione del proponente sarà rivolta principalmente a questo se non reinventiamo da zero il captcha.
        \end{itemize}