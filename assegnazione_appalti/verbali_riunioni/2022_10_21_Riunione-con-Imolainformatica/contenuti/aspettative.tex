\section{Punti del giorno e aspettative}

    \subsection{Visione della bozza architetturale}
    \begin{itemize}
        \item È stata presentata una idea generale degli obbiettivi del progetto.
    \end{itemize}
    \subsection{Definizione precisa di un MVP e degli obiettivi}
    \begin{itemize}
        \item Questo punto verrà trattato dopo aver preso il capitolato.
    \end{itemize}
    \subsection{Discussione delle tecnologie}
    \begin{itemize}
        \item Per database si può utilizzare sia PostgreSQL che NoSQL;
        \item i server sono offerti dall'azienda;
        \item per la comunicazione dei componenti IoT il protocollo MQTT;
        \item ApiRest utilizzando Flask;
        \item React per lo sviluppo della WebApp;
        \item molta flessibilita nei linguaggi di programmazione;
        \item nessun limite imposto.
    \end{itemize}
    \subsection{Tempistiche di rilascio}
    \begin{itemize}
        \item Incontri bisettimanali flessibili.
    \end{itemize}
    \subsection{Nozioni di testing}
    \begin{itemize}
        \item SonarQube per testing del codice e coverage che dovrà superare l'80\%.
    \end{itemize}
    \subsection{Utilizzo Github}
    \begin{itemize}
        \item GitHub come VCS;
        \item continuous Integration con GitHub Actions.
    \end{itemize}
    \subsection{Metodi di comunicazione}
    \begin{itemize}
        \item Gruppo Telegram per mettersi in contatto diretto con l'azienda;
        \item responsabili disponibili anche a contatti diretti tramite mail e telegram.
    \end{itemize}
    \subsection{Informazioni su luci e sensori forniti}
    \begin{itemize}
        \item Utilizzo di led dimmerabili e generici sensori;
        \item il punto critico del progetto sarà la scalabilità dei lampioni.
    \end{itemize}
