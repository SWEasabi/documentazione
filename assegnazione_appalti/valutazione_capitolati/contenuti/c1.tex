\subsection{Capitolato C1: \textit{CAPTCHA: Umano o Sovrumano?}}

\subsubsection{Informazioni generali}

\textbf{Proponente:} \textit{Zucchetti S.p.A.}

\textbf{Descrizione:}
Il capitolato richiede di costruire un sistema di autenticazione in grado di distinguere se un utente è un essere umano oppure un robot.

Per lo sviluppo del sistema si richiede di tenere conto ed analizzare quali possano essere le problematiche di sicurezza legate alla tematica. È richiesto di analizzare le tecnologie di Machine Learning, Intelligenza Artificiale o di qualunque altro tipo che con il tempo possano rendere inefficace il sistema di verifica sviluppato.

\subsubsection{Dominio}

\textbf{Dominio applicativo}

Il capitolato vuole sviluppare una sistema web costituito da una pagina di login e un sistema di autenticazione.

Questo sistema dovrà essere in grado di distinguere un utente umano da un robot.

È richiesto che il sistema sia capace di:
\begin{itemize}
    \item Dimostrare che il captcha non è eludibile;
    \item Comunicare con il server in maniera sicura;
    \item Essere efficente ed efficacie anche in futuro.
\end{itemize}

\textbf{Tecnologie}

Per lo sviluppo dell'applicazione è necessario conoscere e utilizzare:
\begin{itemize}
    \item HTML/CSS/JavaScrip lato client;
    \item Java o PHP lato server.
\end{itemize}

Ma non è fissato nessun requisito obbligatorio sulle stesse.

\subsubsection{Fattori determinanti}
\begin{itemize}
    \item Reinventare un sistema esistente;
    \item Ambiguità nei dettagli;
    \item Capitolato tendenzialmente dispersivo.
\end{itemize}

\subsubsection{Decisione}
L'ambito applicativo del capitolato ci sembra a suo modo interessante, quanto spicca è però il suo essere poco rivolto al futuro essendo la tecnologia captcha datata e poco accessibile. Questo ci fa pensare che la stessa verrà infatti abbandonata presto.

Preso in considerazione quando detto \textbf{rifiutiamo questo capitolato}.
