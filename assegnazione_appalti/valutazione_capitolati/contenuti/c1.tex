\subsection{Capitolato C1: \textit{CAPTCHA: Umano o Sovrumano?}}

\subsubsection{Informazioni generali}

\textbf{Proponente:} \textit{Zucchetti S.p.A.}
\subsubsection{Descrizione}
Il capitolato richiede che venga reinventato e costruito un nuovo sistema di captcha in grado di distinguere se un utente è un essere umano o un robot. Il sistema inoltre dovrà tenere contro delle tecnologie di Machine Learning, Intelligenza Artificiale o qualunque altro tipo che possano con il tempo
rendere inefficace il sistema di verifica.  %tipo test di turing

\subsubsection{Dominio}

\textbf{Dominio applicativo}

Il capitolato vuole sviluppare una applicazione web costituita da una pagina di login che presenti un sistema in grado di distinguere un utente umano da un bot.
Il sistema dovrà quindi:
\begin{itemize}
    \item Dimostrare che il captcha non è eludibile;
    \item Comunicare direttamente con il server;
    \item Essere efficente anche in futuro.
\end{itemize}
% descrivere domino applicativo

\textbf{Tecnologie}

Per lo sviluppo dell'applicazione è necessario conoscere e usare:
\begin{itemize}
    \item HTML/CSS/JavaScrip lato client;
    \item Java o PHP lato server.
\end{itemize}

%divertitevi con le tecnologie

\subsubsection{Fattori determinanti}
\begin{itemize}
    \item Reinventare un sistema esistente
    \item Ambiguità nei dettagli
    \item Capitolato troppo lungo e dispersivo
\end{itemize}

\subsubsection{Decisione}
L'ambito applicativo del capitolato ci sembra a suo modo interessante ma soprattutto poco rivolto al futuro essendo la tecnologia captcha datata e poco accessibile, quindi destinata a essere abbandonata. Questi sono stati i fattori decisionali che ci hanno portato a rifiutare questo progetto.
%inserire wrap-up scelta con giustificazione (rifiutato/troppo complesso/considerato ma non scelto)
