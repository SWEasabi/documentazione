\subsection{Capitolato C6: \textit{Showroom 3D}}

\subsubsection{Informazioni generali}

\textbf{Proponente:} \textit{Sanmarco Informatica S.p.A.}
\subsubsection{Descrizione}
Il capitolato richiede la creazione di uno showroom virtuale  per presentare i prodotti, farsi conoscere e vendere alla clientela, avvalendosi dell’ambiente e dell’esperienza immersiva offerta.  %tipo test di turing

\subsubsection{Dominio}

\textbf{Dominio applicativo}

Il capitolato vuole che venga creato un ambiente tridimensionale navigabile con mouse e tastiera, dove poter esporre una collezione di elementi vendibili.
% descrivere domino applicativo

\textbf{Tecnologie}

Per lo sviluppo dell'applicazione è necessario viene consigliato l'uso di Three.js o in alternativa Unity o Unreal Engine.

%divertitevi con le tecnologie

\subsubsection{Fattori determinanti}
\begin{itemize}
    \item Complessità elevata del progetto
    \item Ambienti di sviluppo
    \item Capitolato troppo lungo
\end{itemize}

\subsubsection{Decisione}
L'ambito applicativo del capitolato ci è sembrato allo stesso tempo affascinante e complesso, la sua difficoltà e la poca esperienza con le tecnologie da utilizzare ci ha portato a decidere di rifiutare il progetto.
%inserire wrap-up scelta con giustificazione (rifiutato/troppo complesso/considerato ma non scelto)