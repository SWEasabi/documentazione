\subsection{Capitolato C5: \textit{SmartLog}}

\subsubsection{Informazioni generali}

\textbf{Proponente:} \textit{Socomec S.r.l.}

\textbf{Descrizione:} Viene proposta la creazione di due applicazioni per visualizzare ed analizzare dei file di log in ambito alimentazione industriale e UPS.

Le applicazioni richieste hanno lo scopo di estrapolare informazioni statistiche dai suddetti log e vorrebbero migliorare la lettura degli stessi permettendo una visione più ampia delle problematiche agli impianti.

\subsubsection{Dominio}

\textbf{Dominio applicativo}

Il capitolato richiede che vengano sviluppate due applicazioni, dove sarà possibile caricare uno o più file di log da analizzare. Non viene vincolata la scelta architetturale e di tecnologie ma viene consigliato un approccio web.

\textbf{Tecnologie}

Per lo sviluppo delle due applicazioni è necessario conoscere e usare:
\begin{itemize}
    \item HTML/CSS/JavaScrip lato client per eventuale app web;
    \item tecnologia di nostra scelta per la produzione delle applicazioni;
    \item viene consigliato Python per l'analisi dei dati.
\end{itemize}

\subsubsection{Fattori determinanti}
\begin{itemize}
    \item Chiarezza e dettaglio nei requisiti;
    \item competenze richieste vicine a quelle del gruppo;
    \item flessibilità implementativa;
    \item a livello di obiettivi meno stimolante del capitolato scelto.
\end{itemize}

\subsubsection{Decisione}

Questo capitolato \textbf{non è stato la nostra prima scelta}. L'ambito applicativo non ci ha stimolato quanto le altre proposte ma la sua chiarezza e adeguata sfida ci ha interessato e fatto valutare con particolare attenzione lo stesso.