\subsection{Capitolato C2: \textit{Lumos Minima}}
\subsubsection{Informazioni generali}
\textbf{Proponente:} \textit{Imola Informatica}
\textbf{Descrizione:}
Il capitolato richiede lo sviluppo di un sistema per ottimizzare l'illuminazione pubblica permettendo ai gestori di regolare l'intensità della luce. Questo può avvenire sia tramite operatore (gestione manuale)
che tramite i dati di un sensore. E' inoltre prevista una fase di inserimento e rilevazione di guasti.

\subsubsection{Dominio}

\textbf{Dominio applicativo}

Il capitolato vuole sviluppare una applicazione web responsive dotata delle seguenti funzioni:
\begin{itemize}
    \item Login e Logout di un'operatore;
    \item Collegamento di un impianto ai dati derivanti da un sensore;
    \item Gestione manuale di un impianto luminoso;
    \item Aumento o riduzione globale dell'intensità luminosa (da operatore o tramite dati di sensore);
    \item Aumento o riduzione locale dell'intensità luminosa (da operatore o tramite dati di sensore);
    \item Permettere all'operatore di inserire manualmente un guasto
    \item Rilevazione automatica di anomalie tramite sensore
\end{itemize}

\textbf{Tecnologie}

Per lo sviluppo dell'applicazione è necessario conoscere e usare:
\begin{itemize}
    \item % MiA dal documento
    \item % MiA dal documento
\end{itemize}



\subsubsection{Fattori determinanti}
\begin{itemize}
    \item Chiarezza del capitolato e disponibilità del referente
    \item Progetto green
    \item Competenze richieste vicine a quelle del gruppo
\end{itemize}

\subsubsection{Decisione}
Il capitolato ci ha fin da subito colpito molto per la sua tematica stimolante e improntata al futuro ma utile già dal presente. Si prefissa di risolvere tematiche legate al risparmio energetico.

Inoltre la libertà lasciatoci a livello di organizzazione e tecnologie e una 
percepita grande disponibilità da parte dell'azienda proponente a lavorare con noi ci ha convinti a scegliere questo capitolato.
