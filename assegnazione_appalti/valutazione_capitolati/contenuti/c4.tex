\subsection{Capitolato C4: \textit{Piattaforma localizzazione testi}}

\subsubsection{Informazioni generali}

\textbf{Proponente:} \textit{AZero12}

\textbf{Descrizione:} Viene richiesta una piattaforma che consenta di centralizzare e riutilizzare le localizzazioni utilizzate da varie aziende in diverse applicazioni. 
\subsubsection{Dominio}

\textbf{Dominio applicativo}

È richiesta la progettazione e messa in opera di un'architettura che consenta una gestione semplificata delle localizzazioni.

Per il proponente è importante la suddivisione dei ruoli all'interno dell'applicazione.

\textbf{Tecnologie}

Tecnologicamente è richiesta la costruzione di una piattaforma strutturata a microservizi. Altro requisito è la progettazione ed implementazione di una libreria di connessione per le applicazioni finali in almeno una delle seguenti tecnologie:
\begin{itemize}
    \item Typescript;
    \item Kotlin;
    \item Swift.
\end{itemize}

\subsubsection{Fattori determinanti}
\begin{itemize}
    \item Proposta ben strutturata;
    \item proponenti disponibili;
    \item idea attuale.
\end{itemize}

\subsubsection{Decisione}
Questo capitolato non è stato la nostra prima scelta. Il proponente ci ha sicuramente colpito molto, ma l'ambito applicativo è risultato meno interessante di altri capitolati a noi proposti.
