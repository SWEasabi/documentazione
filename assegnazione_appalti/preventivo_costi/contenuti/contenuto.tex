\section{Impegno orario}

Il tredicesimo gruppo composto da 7 studenti si impegna in toto a contribuire per ore \textbf{95} pro capite allo svolgimento del progetto didattico.

Suddette ore verranno suddivise come segue, tenendo conto dei ruoli definiti.

Viene inoltre applicato un \textbf{10\%} di coefficente di rendimento nel conteggio di ore reali rispetto alle ore produttive.

\begin{center}
    \begin{tabularx}{10cm}{X |l|l}
        \textbf{Ruolo}  & \textbf{Costo orario} & \textbf{Ore per ruolo} \\
        \hline
        Responsabile    & §30                   & 60                     \\
        Amministratore  & §20                   & 67                     \\
        Analista        & §25                   & 133                    \\
        Progettista     & §25                   & 153                    \\
        Programmatore   & §15                   & 152                    \\
        Verificatore    & §15                   & 100                    \\
        \hline
        \textbf{Totale} & §14064,75             & 665
    \end{tabularx}
\end{center}

La distribuzione per ruolo è quindi come segue.

\begin{minipage}[c][][c]{0.49\linewidth}
    \includegraphics[width=\textwidth]{immagini/Costo ruolo.png}
\end{minipage}
\begin{minipage}[c][][c]{0.49\linewidth}
    \includegraphics[width=\textwidth]{immagini/Ore per ruolo.png}
\end{minipage}

Ogni ruolo avrà inoltre rotazione costante e sistematica ma le ore preventivate per ogni studente si attestano a questi valori:

\begin{center}
    \begin{tabularx}{13cm}{X |l l l l l l| X}
        \textbf{Nome}   & \textbf{re} & \textbf{am} & \textbf{an} & \textbf{pro} & \textbf{prog} & \textbf{ve} & \textbf{Totale} \\
        \hline
        Alessandro      & 9           & 8           & 18          & 24           & 20            & 16          & 95              \\
        Davide          & 9           & 10          & 18          & 22           & 23            & 13          & 95              \\
        Giorgio         & 8           & 9           & 20          & 22           & 23            & 13          & 95              \\
        Luca            & 9           & 10          & 18          & 22           & 20            & 16          & 95              \\
        Mattia          & 8           & 10          & 20          & 21           & 22            & 14          & 95              \\
        Michele         & 9           & 10          & 19          & 21           & 22            & 14          & 95              \\
        Samuel          & 8           & 10          & 20          & 21           & 22            & 14          & 95              \\
        \hline
        \textbf{Totale} & 60          & 67          & 133         & 153          & 152           & 100         & 665
    \end{tabularx}
\end{center}

\section{Ruoli e considerazioni}

\paragraph{Responsabile} Dopo attenta analisi il numero di ore di responsabile risulta ridotta in quanto ce ne sarà molto bisogno nelle fasi iniziali ma nel tempo la tendenza sarà quella dell'autogestione.

\paragraph{Amministratore} Sarà una delle figure più importanti nelle fasi di avviamento del progetto in quanto sarà suo onere improntare e controllare che tutte le procedure siano eseguite a regola d'arte e seguendo i corretti standard. Anche questa figura tenderà a sfumare con l'avvenire del progetto.

\paragraph{Analista}L'analista avendo l'onere di spianare la strada al progettista e al programmatore avrà bisogno di molto tempo così da poter rendere più agevole il compito ad essi. Avrà inoltre bisogno di molto tempo per documentare correttamente quali sono i bisogni del cliente.

\paragraph{Progettista} Abbiamo deciso di dedicare al progettista molte ore, poiché un'architettura ben strutturata rende il lavoro del programmatore più semplice e veloce da svolgere.

\paragraph{Programmatore} Sicuramente servià un buon tempo per stendere il codice in modo corretto.

\paragraph{Verificatore} Il verificatore dovrà assicurare molteplici attività di controllo e verifica durante tutto lo svolgimento. Non avrà quindi mai pace fino al compimento del progetto.

\section{Costi}

Il preventivo per la realizzazione del progetto ammonta quindi a §14064,75 che arrotondiamo per eccesso e tenendo a mente possibili imprevisti a \textbf{§14100,00}

\section{Consegna}

La previsione di consegna viene calcolando una media di \textbf{12} ore settimanali dedicate al progetto. Risultano quindi 9 settimane per terminare lo stesso.

Tenendo quindi conto anche di possibili imprevisti, giorni non lavorativi e cause di forza maggiore non ad ora preventivabili, la previsione di consegna chiavi in mano si attesta oggi al \textbf{06 marzo 2023}
