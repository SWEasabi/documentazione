\section{Obiettivi}

L’obiettivo è sviluppare un’applicazione \textit{web responsive} in grado di monitorare e di eseguire le azioni sotto menzionate in un sistema di illuminazione pubblico.

\subsection{Assunti preventivi}

\begin{itemize}
    \item il gestore è in possesso di uno smartphone (Android o iOS) e nelle condizioni di utilizzare un browser web.
\end{itemize}

\subsection{Obiettivi obbligatori}

\begin{itemize}
    \item Rilevamento della presenza di persone in prossimità della fonte luminosa: attraverso sensori eterogenei (bluetooth, IR, videocamere, ecc.) il sistema rileva la presenza di persone in un’area della città gestita dal sistema e attiva le luci di conseguenza.
    \item Aumento/riduzione dell’intensità luminosa: il sistema deve essere in grado di comunicare attraverso un protocollo IoT l’aumento o la riduzione dell’intensità luminosa emessa da uno specifico impianto di illuminazione, permettendo una regolazione singola o per l’intera via/piazza (area coperta dal servizio).
    \item Rilevamento automatico del guasto di un impianto di illuminazione: il sistema deve rilevare ogni eventuale malfunzionamento dell’impianto di illuminazione, notificando il gestore ed aprendo un ticket di assistenza su una piattaforma esterna che permetta di identificare la località del guasto.
    \item Segnalazione manuale del guasto di un impianto di illuminazione: il sistema deve permettere di inserire manualmente un nuovo guasto ad un impianto di illuminazione.
    \item Aumento/riduzione manuale dell’intensità luminosa: il sistema deve permette al gestore di aumentare/ridurre o riportare in modalità automatica l’illuminazione di un singolo impianto luminoso o di un’intera area coperta dal servizio.
    \item Inserimento e gestione di un impianto luminoso: il sistema deve permettere al gestore
    l’inserimento di nuovi impianti luminosi e il loro inserimento o rimozione all’interno di un’area coperta dal servizio.
    \item Aumento o riduzione globale dell’intensità luminosa: il sistema deve permettere al gestore l’aumento o la riduzione globale (su tutti gli impianti controllati dell’intensità luminosa, opzione particolarmente utile in condizioni di crepuscolo o luce lunare particolarmente intensa.
    
\end{itemize}

\subsection{Obiettivi secondari accennati con l'azienda}

\begin{itemize}
    \item modulare la luminosità a seconda della quantità di persone o tipologia di entità;
    \item Rilevamento tipologia guasti (Guasti locali o distribuiti, guasti a sistemi di alimentazione);
    \item Preset di illuminazione per le varie situazioni;
    \item integrazione con previsioni di eventi normali(alba e tramonto) ed anomali come eclissi;
    \item Gestione multi utente ("gestore illuminazione pubblica", "installatore/manutentore", "Verificatore d'impianto" , altro);
    \item Pagina disponibile ai "cittadini" dove questi possano (seguendo 2/3 passaggi, ad esempio caricando un'immagine del guasto) caricare una segnalazione di guasto.
\end{itemize}

\subsection{Ottica di espansione futura}

\begin{itemize}
    \item Integrazioni future con sistemi di alimentazione/UPS vari nella gestione guasti/distribuzione dell'alimentazione;
    \item Api pubblica regolamentata utilizzabile in futuro da pannellistica a led per mostrare ai cittadini i risparmi o altro.
\end{itemize}

\section{Prima stesura architetturale}

Visto l'ambito critico di operazione del prodotto software per noi i capostipiti nello sviluppo di questo progetto saranno: alta modularità, alta scalabilità, alta resilienza e facile estensibilità dello stesso.

L'architettura seguirà quindi i principi base di un sistema a microservizi che tenga conto di tutte queste importanti questioni.

\subsection{User stories / casi d'uso}

1. login e logout di un operatore
2. collegamento di un impianto luminoso ai dati derivanti da un sensore (modalità automatica)
3. gestione manuale di un impianto luminoso
4. aumento o riduzione globale dell’intensità luminosa da parte di un operatore o tramite dati di un
sensore
5. aumento o riduzione locale (per area illuminata) dell’intensità luminosa da parte di un operatore o tramite dati di un sensore
6. Gestione ticketing guasti
7. inserimento e gestione di un impianto luminoso
8. creazione, modifica e rimozione di nuove aree illuminate
9. tracciamento delle intensità luminose di ogni impianto.
10. Rilevamento della presenza in un’area illuminata e aumento automatico dell’intensità
luminosa

\subsection{Utenti}

Molteplici saranno gli utenti che utilizzeranno il sistema.

\begin{center}
    \begin{tabularx}{14cm}{|X|X| X|}
        \hline
        Utente & utilizzi & Tipo di requisito\\
        \hline        
        Semplice cittadino& Può vedere una dashboard relativa all'illuminazione &Aggiuntivo\\
        \hline 
        Gestore dell'illuminazione & Può impostare l'illuminazione& Obbligatorio\\
        \hline 
        Gestore momentaneo &Può impostare l'illuminazione per un periodo limitato di tempo & Aggiuntivo\\
        \hline 
        Installatore/manutentore & Aggiunge nuove sezioni illuminanti, risolve i guasti & Obbligatorio \\ 
        \hline 
        Verificatore di impianto&Gira a controllare periodicamente se ci sono guasti ai corpi illuminanti & Obbligatorio\\
        \hline

    \end{tabularx}
\end{center}


\subsection{Servizi}

\begin{center}
    \begin{tabularx}{14cm}{|X|X| X|}
        \hline
        Servizio & Scopo & Tecnologia\\
        \hline        
        Mqtt& Comunicazione delle componenti IoT  &Mosquitto\\
        \hline 
        Database & Stoccaggio a lungo termine dei dati per un'analisi futura degli stessi per prevenire guasti o risolvere situazioni ricorrenti & Postgres\\
        \hline 
        Coordinatore &Coordinamento e gestione diretta degli apparati illuminanti& Python\\
        \hline 
        ApiREST del sistema d'illuminazione&Api Backend per la webapp e altri utilizzi futuri&Python, Flask\\
        \hline
        Backend/Api Ticketing&Gestisce il sistema di ticketing dei guasti&Python, Flask\\
        \hline
        WebApp&Consente agli utenti(Definiti più avanti) di interfacciarsi con il sistema&VueJs o React\\
        \hline
    \end{tabularx}
\end{center}





\subsection{Gestione di Deploy}

Per il deploy sarà utilizzato **Docker** per consentire, alla bisogna, lo scalare orizzontale del sistema, così da poter gestire più utenti abbattendo i costi.
        
Il sistema sarà \textit{multi-tennant} as a service oppure installabile on premise.

\section{Testing}

Ognuno dei servizi avrà la sua specifica strategia di testing.

I test di ognuno dovranno avere comunque almeno l'80% di code coverage e dovranno essere correlati di report relativamente all'esecuzione degli stessi.
