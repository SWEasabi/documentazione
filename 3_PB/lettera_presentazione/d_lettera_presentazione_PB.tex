\documentclass{article}

\usepackage{fontspec}
\usepackage{geometry}
\usepackage[table]{xcolor}
\usepackage{tabularx}
\usepackage{graphicx}
\usepackage{mathtools}
\usepackage[bottom]{footmisc}
\usepackage[italian]{babel}
\usepackage{hyperref}
\usepackage{titlesec}
\usepackage{listings}
\usepackage{color}
\usepackage{graphicx}

\lstset{ % General setup for the package
    basicstyle=\small\sffamily,
    numbers=left,
     numberstyle=\tiny,
    frame=tb,
    tabsize=4,
    columns=fixed,
    showstringspaces=false,
    showtabs=false,
    keepspaces,
    commentstyle=\color{red},
    keywordstyle=\color{blue}
}

\geometry{
a4paper,
total = {130mm, 240mm},
left = 40mm,
top = 20mm,
}

\setlength{\parindent}{0em}
\setlength{\parskip}{0.7em}

\titlespacing*{\section}{0pt}{0.7em}{0.5em}
\titlespacing*{\subsection}{0pt}{0.7em}{0.5em}

\begin{document}

\begin{center}
    \begin{minipage}{0.49\linewidth}
        \begin{flushleft}

        \begin{minipage}{0.5\linewidth}
        \begin{center}
        \includegraphics[width=1\linewidth]{\gassets global-assets/img/logo_unipd.png}
        
        \normalsize Università di Padova
        
        \end{center}
        \end{minipage}
        \end{flushleft}
    \end{minipage}
    \begin{minipage}{0.49\linewidth}
        \begin{flushright}

        \begin{minipage}{0.5\linewidth}
        \begin{center}

        \Large SWEasabi

        \tiny Green and spicy software

        \includegraphics[width=0.67\linewidth]{\gassets global-assets/img/loghi/SWEasabi_logo.png}
        
        \small sweasabi@gmail.com

        \end{center}
        \end{minipage}
        \end{flushright}
    \end{minipage}

    \vspace{1cm}

    Corso di Ingegneria del Software A.A.:2022/2023
    
    \vspace{2cm}

    \begin{minipage}[t]{0.49\linewidth}
        \begin{flushleft}

        \begin{minipage}[t]{0.8\linewidth}
        \begin{flushleft}
        
            \normalsize
            \textit{Destinatario:}
            \vspace{0.5cm}

            Prof. Vardanega Tullio,\\
            Prof. Cardin Riccardo,\\
            Università degli Studi di Padova,\\
            Dipartimento di Matematica,\\
            Via Trieste, 63\\
            35121 Padova\\        
        \end{flushleft}
        \end{minipage}
        \end{flushleft}
    \end{minipage}
    \begin{minipage}[t]{0.49\linewidth}
        \begin{flushright}

        \begin{minipage}[t]{1\linewidth}
            
        \begin{flushright}
        \normalsize
        \textit{Mittente:}
        \vspace{0.5cm}

        Il gruppo \textit{SWEasabi} composto di:
            \begin{tabularx}{0.87\linewidth}{l | X}
                Michele Bonavigo & 1216752 \\
                Mattia Casarotto & 1221948 \\
                Alessandro Massarenti & 1204684 \\
                Samuel Peron & 1225423 \\
                Luca Pierobon & 2008649 \\
                Davide Romano & 2008652 \\
                Giorgio Zarantonello & 1097629 
            \end{tabularx}
        \end{flushright}
        \end{minipage}
        \end{flushright}
    \end{minipage}
    
    \normalsize
\end{center}

\vspace{1,5cm}

Egregio Prof. Vardanega Tullio,\\
Egregio Prof. Cardin Riccardo,\\

con la presente il gruppo \textit{SWEasabi} intende comunicarLe ufficialmente la propria \textbf{candidatura alla presentazione della Revisione di \textit{Product Baseline}} per il progetto da Voi commissionato, denominato \textbf{Lumos Minima} e proposto dall'azienda \textit{Imola Informatica S.p.A.}.

Il gruppo conferma il proprio impegno consegnandoVi la documentazione di seguito allegata.

Cordialmente,

\begin{flushright}
\textit{I 7 componenti di SWEasabi}
\end{flushright}

In allegato alla presente candidatura rendiamo disponibile tutta la documentazione coadiuvante all'indirizzo web:

\begin{center}
    \href{https://sweasabi.github.io/documentazione/3\_PB/}{\textit{https://sweasabi.github.io/documentazione/3\_PB/}}
    %TODO: Va messo il link verso /PB non solo documentazione
\end{center}

In alternativa si rende disponibile il repository pubblico:

\begin{center}
    \href{https://github.com/SWEasabi/documentazione/tree/main/3\_PB/}{\textit{https://github.com/SWEasabi/documentazione/tree/main/3\_PB/}}

\end{center}

La documentazione allegata si compone di:
\begin{itemize}
    %TODO: Correggere le versioni prima di consegnare
    \item \textbf{Analisi dei Requisiti (v1.0.2):} \textit{analisi-dei-requisiti.pdf}
    \item \textbf{Norme di Progetto (v2.0.0):} \textit{norme-di-progetto.pdf}
    \item \textbf{Piano di Progetto (v2.0.1):} \textit{piano-di-progetto.pdf}
    \item \textbf{Piano di Qualifica (v2.0.0):} \textit{piano-di-qualifica.pdf}
    \item \textbf{Specifica architetturale (v1.0.0):} \textit{specifica-architetturale.pdf}
    \item \textbf{Specifiche delle basi di dati:} \textit{appendice al documento di Specifica architetturale}
    \item \textbf{Glossario (v1.0.1):} \textit{glossario.pdf}
    \item \textbf{Manuale utente (v1.0.0):} \textit{manuale-utente.pdf}
    \item \textbf{Verbali interni ed esterni}
\end{itemize}


\end{document}
