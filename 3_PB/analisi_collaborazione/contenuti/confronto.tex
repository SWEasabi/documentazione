\section{Analisi delle problematiche e possibili soluzioni}

\subsection{Sessione estiva}

La sessione estiva è stata fonte di un gran rallentamento. 
Principalmente, come nella sessione precedente, lo studio per gli esami ha portato ad una riduzione del tempo dedicato al progetto. Per alcuni dei membri ha portato anche al fermo totale delle attività.

Interessante, di questa fase è l'effetto a cascata che ha anche contribuito all'emergere delle altre problematiche discusse. Rallentare e fermarsi è stato fatale e ha permesso di dimenticare la routine costruita con le \textit{norme di progetto}.

\subsection{Mancanza di comunicazione}

La comunicazione si è rivelata prolematica sia per questioni di mancata comunicazione e tracciamento degli avanzamenti, sia per questioni di uniformità comunicativa.

L'uniformità comunicativa viene discussa maggiormenete nella sezione successiva.

La mancata comunicazione e tracciamento degli avanzamenti è, più in dettaglio, quello che viene definito come "fare i sottomarini". Un esempio di questa pratica è stato quando alcuni membri hanno iniziato a lavorare su un documento notificando agli altri lo stato di avanzamento in ritardo.

Questa mancanza di comunicazione, sarebbe comunque stata risolvibile utilizzando più frequentemente l'ITS di GitHub, rendendo asincrone le attività programmate.

\subsection{Scarsa applicazione delle norme di progetto}

La scarsa applicazione delle \textit{norme di progetto} è stata la causa principale del rallentamento.

Il gruppo non è stato particolarmente abile nell'unificare le comunicazioni su un mezzo comune. Da \textit{norme di progetto} il mezzo in oggetto è l'ITS di GitHub che consente di avere un tracciamento delle attività e dei problemi ed inoltre di avere un canale di comunicazione separato per ognuno di essi. Il gruppo Telegram, con l'avazare del progetto, ha progressivamente emulato le funzionalità dell'ITS fino a diventare il mezzo principale di comunicazione, senza però diventarlo in forma ufficiale. Questo ha portato ad una frammentazione delle informazioni e ha compromesso la tracciabilità dell'avanzamento delle attività. Di conseguenza, la chat si è rivelata congestionata e poco fruibile a causa del volume eccessivo di comunicazioni.

Una possibile soluzione sarebbe stato modificare le norme di progetto per rendere più agevole la comunicazione, riducendo le lentezze burocratiche dell'ITS.

Un esempio di proposta poteva essere continuare ad usare l'ITS per tracciare e utilizzare i messaggi privati su Telegram per notificare di un avanzamento.
