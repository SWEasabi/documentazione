\section{Confronto}

In seguito ai problemi segnalati nella sezione precedente, il gruppo ha svolto un confronto, con la presenza di tutti i membri, al fine di chiarire che cosa ha portato a questi problemi e coordinarsi sulle attività da fare al fine di poter effettuare la revisione semaforica prima e la PB poi.

Oltre agli impegni personali di alcuni membri del gruppo, cosa per cui non si è potuto fare molto, la problematica principale è stata la mancanza di un'uniformità comunicativa.
Alcuni membri hanno utilizzato l'ITS di GitHub mentre altri hanno utilizzato la chat di gruppo di Telegram come mezzo di comunicazione principale.

Quest'ultima spesso non è stata vista da alcuni membri causando, involontariamente, dei rallentamenti. Un esempio è stato quando i progettisti non hanno ricevuto un feedback su quanto pensato per i microservizi. Questo ha portato la stima per la PB ad un primo spostamento in avanti.

Per quanto riguarda l'uniformità comunicativa invece, sin dall'inizio del progetto i vari responsabili hanno raccomandato di utilizzare principalmente l'ITS di GitHub e come estrema ratio la chat di Telegram, inviti che però, sebbene nell'ultimo periodo non siano stati ribaditi, non sono stati ascoltati da alcuni membri del gruppo.

Questo problema sarebbe potuto essere tamponato sul nascere scrivendo in privato ai membri interessati, sempre su Telegram. La chat di gruppo non è lo strumento adatto a gestire grosse moli di messaggi, senza categorie e senza sistemi di tracciamento. Il gruppo Telegram è risultato particolarmente intasato per la mole di lavoro.
La metodologia appena esposta non è però stata utilizzata particolarmente, ed è apparsa come soluzione solo in secondo luogo.